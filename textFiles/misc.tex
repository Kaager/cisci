\section{Miscellaneous}
\subsection{Stop dem Lookups (Disable DNS Lookup)}
(config)\# no ip domain-lookup

\subsection{If you fucked up}
\# erase startup-config \\
\# delete vlan.dat\footnote{If present, check with ``dir flash:''}\\
\# reload

\subsection{Disable CPD (Cisco Descovery Protocol)}
(config)\# no cdp run

\subsection{Logging Synchronous}
(config)\# line console 0 \\
(config-line)\# logging synchronous

\subsection{Set Console Time to Never Timeout}
(config)\# line console 0 \\
(config-line)\# exec-timeout 0 0\footnote{first 0 is for minutes, second is for seconds}

\section{Backing up and restoring TFTP (Trivial File Transfer Protocol)}
\# copy running-config/startup-config tftp\\
Enter IP address of the target server:\\
Enter desired name for the file:\\
Press ‘Enter’ to confirm\vspace{11pt}\\
\# copy tftp running-config/startup-config \\
Enter the IP address of the host where the configuration file is \\stored:\\
Enter the name to assign to the configuration file: \\
Press ‘Enter’ to confirm

\section{Configuring SSH}
\textrm{The device needs an unique hostname}\\
(config)\# ip domain-name \textit{name.something}\\
(config)\# crypto key generate rsa\footnote{Enter desired key length (1024)}\\
(config)\# ip ssh version 2\\
(config)\# username \textit{name} secret \textit{secret\_password}\vspace{11pt}\\
(config)\# ip ssh time-out 60\\
(config)\# ip ssh authentication-retries 2\vspace{11pt}\\
(config)\# line vty 0 15\\
(config-line)\# login local\footnote{Remote login now looks for local user for authentication}\\
(config-line)\# transport input ssh\footnote{Only able to remote via SSH}

\subsection{Delete RSA key pair and disables SSH server}
(config)\# crypto key zerosize rsa