\section{Single Area OSPF}
\subsection{IPv4}
\subsubsection*{Enter and enable OSPF with}
(config)\# \textbf{router ospf} \textit{process\_id}\vspace{4pt}\\
\textrm{The \texttt{\textit{process-id}} value represents a number between 1 and 65,535 and is selected by the network administrator. The \texttt{\textit{process-id}} value is locally significant, which means that it does not have to be the same value on the other OSPF routers to establish adjacencies with those neighbors.}
\subsubsection*{Give the router an ID}
(config-router)\# \textbf{router-id} \textit{rid}\vspace{4pt}\\
\textrm{The rid value is any 32-bit value expressed as an IPv4 address, e.g.} (config-router)\# router-id 1.1.1.1
\subsubsection*{If the router already has a name, you might have to clear the OSPF process before you can change it}
(config-router)\# router-id 1.1.1.1\vspace{11pt}\\
\%\%\% Some error message\vspace{11pt}\\
\# clear ip ospf process\\
\textrm{Press \texttt{y} to the prompt}\vspace{11pt}\\
show ip protocols | section Router ID -> to see the change

\subsubsection*{Using a Loopback Interface as the Router ID}
(config)\# interface loopback 0\\
(config-if)\# ip address 1.1.1.1 255.255.255.255\\
(config-if)\# end\\
\textrm{The IPv4 address of the loopback interface should be configured using a 32-bit subnet mask (255.255.255.255). This effectively creates a host route. A 32-bit host route does not get advertised as a route to other OSPF routers. 
The example displays how to configure a loopback interface with a host route on R1. R1 uses the host route as its router ID, assuming there is no router ID explicitly configured or previously learned.
}
\subsubsection*{Enabling OSPF on interfaces}
\textrm{The \texttt{network} command determines which interfaces participate in the routing process for an OSPF area. Any interfaces on the router that match the network address in the \texttt{network} command are enabled to send and receive OSPF packets. As a result, the network (or subnet) addresses for the interface is included in OSPF routing updates. The basic command syntax is}\vspace{8pt}\\
(config-router)\#\textbf{network} \textit{network\_address wildcard\_mask} \textbf{area} \textit{area\_id}\vspace{11pt}\\
\textrm{The \texttt{\textbf{area} \textit{area\_id}} syntax refers to the OSPF area. When configuring single-area OSPF, the \texttt{\textbf{network}} command must be configured with the same \texttt{area\_id} value on all routers. Although any area ID can be used, it is good practice to use an area ID of 0 with single-area OSPF. This convention makes it easier if the network is later altered to support multiarea OSPF.}\vspace{8pt}\\
(config)\# router ospf 10\\
(config-router)\# network 172.16.1.0 0.0.0.255 area 0\\
(config-router)\# network 172.16.3.0 0.0.0.3 area 0\\
(config-router)\# network 192.168.10.4 0.0.0.3 area 0\vspace{11pt}\\
\textrm{As an alternative, OSPFv2 can be enabled using the \texttt{\textbf{network} \textit{int\_ip\_address} \\\textbf{0.0.0.0} \textbf{area} \textit{area\_id}} router configuration mode command.\\
Below you can see an example of specifying the interface IPv4 address with a quad 0 wildcard mask. Entering \texttt{network 172.16.3.1 0.0.0.0 area 0} on R1 tells the router to enable interface Serial0/0/0 for the routing process. As a result, the OSPFv2 process will advertise the network that is on this interface (172.16.3.0/30).\\
The advantage of specifying the interface is that the wildcard mask calculation is not necessary. OSPFv2 uses the interface address and subnet mask to determine the network to advertise.\\
\textbf{Note:} This method is not really used by Cisco.}\vspace{8pt}\\
(config)\# router ospf 10\\
(config-router)\# network 172.16.1.1 0.0.0.0 area 0\\
(config-router)\# network 172.16.3.1 0.0.0.0 area 0\\
(config-router)\# network 192.168.10.5 0.0.0.0 area 0
\subsubsection*{Propagate the default route}
(config-router)\# \textbf{default-information originate}
\subsubsection*{Passive Interface}
(config-router)\#\textbf{ passive-interface} \textit{interface\_id}
\subsubsection*{Adjusting the interface bandwidth}
(config)\# \textbf{interface} \textit{interface\_id}\\
(config-if)\# \textbf{bandwidth} \textit{kilobits}
\subsubsection*{Manually setting the OSPF cost}
(config)\# \textbf{interface} \textit{interface\_id}\\
(config-if)\# \textbf{ip ospf cost} \textit{value}
\subsubsection*{Some show commands}
show ip ospf neighbor\\
show ip protocols\\
show ip ospf\\
show ip ospf interface\\
show ip ospf interface brief\\
show ip ospf interface serial 0/0/1

\subsection{IPv6}
\textrm{\textbf{Enable IPv6 routing}}\\
(config)\# ipv6 unicast-routing
\subsubsection*{Router ID}
(config)\# \textbf{ipv6 router ospf} \textit{process\_id}\\
(config-rtr)\# \textbf{router-id} \textit{rid}
\subsubsection*{Enable OSPFv3 on the interfaces that contain the networks that you want to be part of the OSPF area}
(config-if)\# \textbf{ipv6 ospf} \textit{process\_id} \textbf{area} \textit{area\_id}
\subsubsection*{Prevent routing updates to be send across an interface}
(config)\# \textbf{ipv6 router ospf} \textit{process\_id}\\
(config-rtr)\# \textbf{passive-interface} \textit{interface\_id}
\subsubsection*{Propagate the default route}
(config-rtr)\# \textbf{redistribute static}
\subsubsection*{Some show commands}
show ipv6 ospf neighbour\\
show ipv6 protocols\\
show ipv6 ospf\\
show ipv6 ospf interface\\
show ipv6 ospf interface brief\\
show ipv6 ospf interface serial 0/0/1\\
show ipv6 route ospf
